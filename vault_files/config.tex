% hyperlinks
\usepackage{hyperref}
\usepackage{tabu}
\usepackage{algorithm}
\usepackage[noend]{algpseudocode}
\usepackage{amsmath}
\usepackage{multicol}
\usepackage{verbatim}
\usepackage{graphicx}
\usepackage{rotating}
\usepackage{cancel}
\usepackage{soul}
\usepackage{titlesec}
\usepackage{physics}
\usepackage{pdflscape}
\usepackage{polyglossia} %This is the package that gives access to fonts.
% for theorems
\usepackage{amsthm}
\usepackage{ amssymb }
\usepackage{wrapfig}
\usepackage{tikz}
\usepackage{tikz-network}
\usetikzlibrary{arrows.meta}
\usetikzlibrary{matrix}
\usetikzlibrary{positioning,chains,fit,shapes,calc}
\usetikzlibrary{calc,intersections,through,hobby}
\usetikzlibrary{mindmap}
\usetikzlibrary {angles,quotes} 

\usepackage[many]{tcolorbox}
\usepackage{subcaption}
\usetikzlibrary{trees}
   	% for COLORED BOXES (tikz and xcolor included)
\usepackage{mathspec} 			    % for FONTS
\usepackage{setspace}               % for LINE SPACING
\usepackage{multicol}     
\usepackage{pgfplots}
\pgfplotsset{compat=1.18} 

\usepackage{pdfpages}
% \usepackage[T1]{fontenc}

% Used to include part of tex files in others
\usepackage{catchfilebetweentags}
% A FUCKING BRILLIANT Patch to fix issue of catchfilebetweentags: it removes newlines...
\makeatletter
\def\CatchFBT@Fin@l#1[#2]{%
   \begingroup
      %\endlinechar\m@ne % <- this is the guilty party
      \makeatletter #2%
      \scantokens\expandafter{%
         \expandafter\CatchFBT@tok\expandafter{\the\CatchFBT@tok}}%
      \CatchFBT@IsAToken{#1}
         {\global#1\expandafter{\the\CatchFBT@tok}}
         {\xdef#1{\the\CatchFBT@tok}}%
      \ifx\CatchFBT@tok#1\else\global\CatchFBT@tok{}\fi
   \endgroup
}% \CatchFBT@Final
\makeatother


% good af for maths


% Change bullet types in itemize
\usepackage{enumitem}
\usepackage{pifont}

% Headers
\usepackage[headheight=12mm,headsep=5mm]{geometry}
\usepackage{fancyhdr}
\pagestyle{fancy}

% Table
\usepackage{amsfonts}
\usepackage{booktabs}
\usepackage{siunitx}
\usepackage{parskip}
\usepackage{polyglossia}
% These two packages
\usepackage{etoolbox}
\usepackage{fontspec}
\setmainfont{David CLM}
\newfontfamily\quotefont{Cardo}
\AtBeginEnvironment{quote}{\quotefont}
\usepackage{tabularx}

\setdefaultlanguage{hebrew}  %this makes the titles RTL, switch these if you write mostly in English
\setotherlanguage{english}  
\newfontfamily\hebrewfont[Script=Hebrew]{David CLM} %This font will work in Overleaf, Linux and Mac installs, if you're running windows you'd need to pick the right ones
\newfontfamily\hebrewfontsf[Script=Hebrew]{David CLM}
\newfontfamily\englishfont{Latin Modern Roman}
% \newfontfamily\hebrewfontrm[Script=Hebrew]{Cardo}

% \setsansfont[Script=Hebrew]{Arial}
% \setmonofont{Arial}


\usepackage{tabularx,booktabs,multirow} % all about tables: booktabs gives the standard formatting, multirow allows cells to span over multiple rows.
\theoremstyle{definition}
% % Define the Hebrew theorem environment
\newtheorem{theorem}{משפט}
\newtheorem{definition}{הגדרה}
% \crefname{theorem}{משפט}{משפטים}
% \crefname{deinition}{הגדרה}{הגדרות}
\newtheorem{claim}{טענה}
% \crefname{claim}{טענה}{טענות}


 \newcommand{\problemdef}[3]{
  \begin{center}
    \begin{minipage}{0.7\textwidth}
      \noindent
      \fontspec{Miriam Mono CLM}
      #1

      \vspace{2pt}
      \setlength{\tabcolsep}{3pt}
      \begin{tabularx}{\textwidth}{lX lX}
        \textbf{קלט:} 	& #2 \\
        \textbf{שאלה:} 	& #3
      \end{tabularx}
    \end{minipage}
  \end{center}
}



\newcommand{\NP}{\textenglish{\textnormal{\textsf{NP}}}}
\newcommand{\PP}{\textenglish{\textnormal{\textsf{P}}}}
% \newcommand{\NP}{NP}
% \newcommand{\PP}{P}
\newcommand{\bigO}{\mathcal{O}}

\usepackage[realmainfile]{currfile}

% Compare Strings
\newcommand{\StrCompare}[2]{%
  \ifthenelse{\equal{\string \detokenize{#1}}{\string \detokenize{#2}}}
    {TRUE}
    {FALSE}%
}{}

% substrings
\usepackage{stringstrings}

% get unit number
\newcommand{\documentNumber}[0]{\substring{\currfilebase}{3}{4}\par}

% colors
\definecolor{LightCyan}{rgb}{0.88,1,1}

\renewcommand{\tightlist}{}%





% CREATE RANDOM INTS
% \usepackage{pgf}
% % \pgfmathsetseed{\number\randomseed} % to ensure that it is randomized 
% \newcommand{\randomnum}[1]{% 
% \pgfmathsetmacro{\a}{1}%
% \pgfmathsetmacro{\b}{int(#1)}% 
% \pgfmathsetmacro{\thenum}{int(random(\a,\b))}%
% \thenum%
% }%
% \randomnum{10}
\usepackage{esint}
\usepackage{fontawesome}
% Champions Tip
\usepackage{stackengine}
\usepackage{scalerel}
\usepackage[dvipsnames]{xcolor}
\usepackage{amssymb}
\newcommand\champTip[1][2ex]{%
  \scaleto{
  \stackengine{-0.1pt}{\color{white}\tiny\bfseries !}{\scalebox{1.}[.9]{%
  \color{red}$\blacktriangle$}}{O}{c}{F}{F}{L}
  }{#1}%
}
\newfontfamily\bulletfont{Latin Modern Roman}
\renewcommand*\labelitemi{\bulletfont\textbullet}
\definecolor{main}{HTML}{5989cf}    % setting main color to be used
\definecolor{sub}{HTML}{cde4ff} 
\definecolor{red}{HTML}{e60000}
\definecolor{green}{HTML}{44cc00}
\definecolor{blue}{HTML}{0066ff}
\newcommand{\etext}[1]{\text{\foreignlanguage{english}{#1}}}


\newtcolorbox{boxH}{
    colback = sub, 
    colframe = main, 
    boxrule = 0pt, 
    rightrule = 6pt % left rule weight
    }

\newtcolorbox{RedBox}{
    colback = red!5!white, 
    colframe = red, 
    boxrule = 0pt, 
    rightrule = 6pt % left rule weight
    }
    
\newtcolorbox{GreenBox}{
    colback = green!5!white, 
    colframe = green, 
    boxrule = 0pt, 
    rightrule = 6pt % left rule weight
    }
\newtcolorbox{BlueBox}{
    colback = blue!5!white, 
    colframe = blue, 
    boxrule = 0pt, 
    rightrule = 6pt % left rule weight
    }
\newtcolorbox{YellowBox}{
    colback = yellow!5!white, 
    colframe = yellow, 
    boxrule = 0pt, 
    rightrule = 6pt % left rule weight
    }

\newtcolorbox{PurpleBox}{
    colback = violet!3!white, 
    colframe =  violet!40!white, 
    boxrule = 0pt, 
    rightrule = 6pt % left rule weight
    }

\newcommand{\yellowb}[1]{\begin{YellowBox}#1\end{YellowBox}}
\newcommand{\purpleb}[1]{\begin{PurpleBox}#1\end{PurpleBox}}
\newcommand{\Blueb}[1]{\begin{BlueBox}#1\end{BlueBox}}
\newcommand{\Greenb}[1]{\begin{GreenBox}#1\end{GreenBox}}
\newcommand{\redb}[1]{\begin{RedBox}#1\end{RedBox}}

\newtcolorbox{boxA}{
    fontupper = \bf,
    boxrule = 1.5pt,
    colframe = black % frame color
}

\newtcolorbox{boxB}{
    fontupper = \bf\color{main}, % font color
    boxrule = 1.5pt,
    colframe = main,
    rounded corners,
    arc = 5pt   % corners roundness
}

\newtcolorbox{boxC}{
     boxrule = 0pt,  % no borders
    rightrule = 2.5pt
}

\newtcolorbox{boxD}{
    colback = sub, 
    colframe = main, 
    boxrule = 0pt, 
    toprule = 3pt, % top rule weight
    bottomrule = 3pt % bottom rule weight
}

\newtcolorbox{boxE}{
    enhanced,
    boxrule = 0pt, % clearing the default rule
    borderline = {0.75pt}{0pt}{main}, % outer line
    borderline = {0.75pt}{2pt}{sub} % inner line
}

\newtcolorbox{boxF}{
    colback = sub,
    enhanced,
    boxrule = 1.5pt, 
    colframe = white, % making the base for dash line
    borderline = {1.5pt}{0pt}{main, dashed} % add "dashed" for dashed line
}

\newtcolorbox{boxG}{
    enhanced,
    boxrule = 0pt,
    colback = sub,
    borderline west = {1pt}{0pt}{main}, 
    borderline west = {0.75pt}{2pt}{main}, 
    borderline east = {1pt}{0pt}{main}, 
    borderline east = {0.75pt}{2pt}{main}
}

\newtcolorbox{boxI}{
    colback = sub, 
    colframe = main, 
    boxrule = 0pt, 
    toprule = 6pt % top rule weight
}

\newtcolorbox{boxJ}{
    sharpish corners, % better drop shadow
    colback = sub, 
    colframe = main, 
    boxrule = 0pt, 
    toprule = 4.5pt, % top rule weight
    enhanced,
    fuzzy shadow = {0pt}{-2pt}{-0.5pt}{0.5pt}{black!35} % {xshift}{yshift}{offset}{step}{options} 
}

\newtcolorbox{boxK}{
    sharpish corners, % better drop shadow
    boxrule = 0pt,
    toprule = 4.5pt, % top rule weight
    enhanced,
    fuzzy shadow = {0pt}{-2pt}{-0.5pt}{0.5pt}{black!35} % {xshift}{yshift}{offset}{step}{options} 
}

\newtcolorbox{boxL}{
    fontupper = \color{main},
    rounded corners,
    arc = 6pt,
    colback = sub, 
    colframe = main!50, 
    boxrule = 0pt, 
    bottomrule = 4.5pt 
}

\newtcolorbox{boxM}{
    fontupper = \color{white},
    rounded corners,
    arc = 6pt,
    colback = main!80, 
    colframe = main, 
    boxrule = 0pt, 
    bottomrule = 4.5pt,
    enhanced,
    fuzzy shadow = {0pt}{-3pt}{-0.5pt}{0.5pt}{black!35}
}
\newtcolorbox{boxAN}{
    boxrule = 0pt,
    rightrule = 2.5pt, % top rule weight
    enhanced,
   
}

 \newcommand\independent{\protect\mathpalette{\protect\independenT}{\perp}}
    \def\independenT#1#2{\mathrel{\setbox0\hbox{$#1#2$}%
    \copy0\kern-\wd0\mkern4mu\box0}} 
            
\newcommand{\noin}{\noindent}    
\newcommand{\logit}{\textrm{logit}} 
\newcommand{\var}{\textrm{Var}}
\newcommand{\cov}{\textrm{Cov}} 
\newcommand{\corr}{\textrm{Corr}} 
\newcommand{\N}{\mathcal{N}}
\newcommand{\Bern}{\textrm{Bern}}
\newcommand{\Bin}{\textrm{Bin}}
\newcommand{\Beta}{\textrm{Beta}}
\newcommand{\Gam}{\textrm{Gamma}}
\newcommand{\Expo}{\textrm{Expo}}
\newcommand{\Pois}{\textrm{Pois}}
\newcommand{\Unif}{\textrm{Unif}}
\newcommand{\Geom}{\textrm{Geom}}
\newcommand{\NBin}{\textrm{NBin}}
\newcommand{\Hypergeometric}{\textrm{HGeom}}
\newcommand{\HGeom}{\textrm{HGeom}}
\newcommand{\Mult}{\textrm{Mult}}
